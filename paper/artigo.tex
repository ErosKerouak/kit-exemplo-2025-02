% Documento LaTex com o artigo 

% Cabeçalho
%%%%%%%%%%%%%%%%%%%%%%%%%%%%%%%%%%%%%%%

\documentclass{article}

\usepackage[brazil]{babel}
\usepackage{graphicx}
\usepackage[round,authoryear,sort]{natbib}
\usepackage{mathpazo}

\input{paises.tex}




% Corpo
%%%%%%%%%%%%%%%%%%%%%%%%%%%%%%%%%%%%%%%
\begin{document}

\title{Análise de variação de temperatura dos ultimos 5 anos}

\author{Eros Kerouak Cordeira Pareira}

\maketitle 

\begin{abstract}
resumo
\end{abstract}

\section{Introdução}

Meu artigo \citep{cheng2014impact}
Isso foi analisado por \citet{cheng2014impact}

\section{Metodologia}

Ajustamos a uma reta aos 5 ultimos anos dos dados de temperatura media mensal para cada pais 

A equação

\begin{equation}
T(t) = a t + b ,
\label{eq:reta}
\end{equation}

\noindent
onde $T$ é a temperatura, $t$ é o tempo, $a$ é o coeficiente angular e $b$ é o coeficiente linear.

Utilizamos a equação 1 em um código python 

\section{Resultados}

Analisamos os dados 


\begin{figure}[!htb]
	\includegraphics{../figuras/variacao_temperatura.png}
	\caption{
		Variação de temperatura média mensal dos cinco últimos anos.
		a) Países com as cinco maiores variações de temperatura .
	}
	\label{fig:variacao}
\end{figure}

Os resultados da ánalise de variação de temperatura \ref{fig:variacao}
\Paises



\bibliographystyle{apalike}
\bibliography{referencia.bib}
\end{document}